\documentclass{article}


\usepackage[utf8]{inputenc}
\usepackage[margin=1.0in]{geometry}
\usepackage{setspace}
\usepackage{dirtytalk}
\usepackage{blindtext}
\usepackage{lipsum}
\usepackage{amsmath}
\usepackage{amsfonts}
\usepackage{amssymb}
\linespread{1.5}

\title{Project Proposal: Montecarlo simulation with Maxwell's Equations.}
\author{Miguel Angel Gómez Barrera.}
\begin{document}
	\maketitle
\paragraph{}Maxwell's equations are a set of four equations, that have been defined by James Clerk Maxwell more than a century ago, these equations gave us a door to a deeper understanding of the universe: laws of physics, that were derived from characters such as Gauss and Faraday. These equations help us describe electromagnetic phenomena, and it is one of the reasons why we have today amazing technologies such as computers, radio, wireless internet and among others.
\begin{equation}
	\nabla \cdot D = \rho_V
\end{equation}
\paragraph{}The first equation is known as Gauss'law, and dictates also that charges of the same sign repel. $D$ is the electric flux density and $\rho_V$ the electric charge density.
\begin{equation}
	\nabla \cdot B = 0
\end{equation}
\paragraph{} the second one is Gauss law of magnetism, and dictates that magnetic monopoles do not exists, here $B$ is the magnetic field.
\begin{equation}
	\nabla \times E = -\frac{\partial B}{\partial t}
\end{equation}
\paragraph{}Known as Faraday's Law, this equation describes how a magnetic field over time can produce an electric field.
\begin{equation}
	\nabla \times H = \frac{\partial D}{\partial t} + J
\end{equation}
\paragraph{} Ampére's Law, this equation describes how electromagnetic fields can be generated by changes in time of the electric flux density or by an electric current density $J$.
\paragraph{} Electromagnetic pehonema is a too wide topic, and certainly these equations are difficult to solve, several numerical methods have been applied depending on the geometry of the problem, such as finite element and finite difference, I will choose to model one electromagnetic phenomena (which I'm not sure at the moment but here are some topics/ideas):
\begin{itemize}
	\item Antennas: Radiation patterns.
	\item Light: reproduce the result numerically of why is that light is an electromagnetic wave.
	\item Neutrino theory of light: ¿Why they satisfy Maxwell's equations?
\end{itemize}
\paragraph{} I will choose one of topics and use finite difference and montecarlo methods, compare the results, and verify which one is better, simpler and accurate, and in the process learn more about this phenomena.

\nocite{*}

\bibliographystyle{unsrt}
\bibliography{bibliography}

\end{document}